\documentclass{scrartcl}

\usepackage[hidelinks]{hyperref}
\usepackage[none]{hyphenat}

\title{AR/ VR Usability}

\subtitle{COMP 210 - Research Essay}

\author{1506919}

\begin{document}

\maketitle

\abstract{This paper focuses on the usability issues that other papers have found in AR and VR}

\section*{Introduction}

AR/ VR are relatively new, also because of the price and room needed to use them they are not as popular as traditional gaming platforms (e.g. Xbox, PlayStation, PC), therefore studies on usability are important so those that own or try it enjoy the experience.

\section*{A good AR/ VR system}

Hardware interfaces for VR Applications: Evaluation on prototypes\cite{mentzelopoulos2015hardware} states that a good VR system must consider four main points of usability;

1) Full field of vision

2) Tracking position and altitude of players body

3) Tracking of players movement and actions

4) Updating the display with feedback from said players movement and actions.

(Re-) Examination of Multi model Augmented Reality\cite{rosa2016re} dictates a good AR system must include the following abilities;

Run interactively and in real time, also register/ align real and virtual objects with each other.

Studies have found if an individual is interacted with in the virtual world the virtual world feels more real\cite{madary2016real}, this can also aid the usability of AR/ VR depending on the purpose, social interactions can help instruct and advise the player on what to do next.

\section*{AR vs. VR usability}

Leena Venta-Olkkanen, Maaret Posti, Olli Koskenranta and Jonna Hakkila\cite{venta2014investigating} conducted a user interface study with 35 participants used in a field study and 136 online comparing AR and VR mobile applications of the urban environment. The virtual was a 3D model of a street, where as the augmented used a live view of the same street. The UI was information about the street and looked exactly the same in both AR/ VR versions.

They found both in the field study and online survey participants preferred the AR live view of the street rating the usability and clarity higher through VR got better ratings on utility, understand-ability and pleasantness.

From this study it appears that when we use technology for information about real places AR increases the usability and realness.

\section*{VR usability heuristics (including AR)}

The paper Exploitation of Heuristics for Virtual Environments\cite{hvannberg2012exploitation} discusses that the generic usability heuristics were not designed with virtual environments in mind and these new technologies require a more specific heuristic list.

Alistair Sutcliffe and Brian Gault present 12 heuristics\cite{sutcliffe2000heuristic} tailored towards the evaluation of virtual reality usability and presence issues;

1) Natural Engagement: User should feel like they would in a real world

2) Compatibility with the users task and domain: The behaviour of objects should be like their real world behaviour

3) Natural expression of action: Be able to act and explore in a natural manner

4) Close coordination of action and representation: Response time between users movement and update of the virtual environment should not be too delayed

5) Realistic feedback: User actions should be immediately visible

6) Faithful viewpoints: Head movements should change the VR display immediately

7) Navigation and orientation support: Users should be able to return to preset positions

8) Clear entry and exit: Exit and entry should be communicated clearly

9) Consistent departures: Design compromises should be clear and consistent

10) Support for learning: Explanations to encourage learning of virtual environments

11) Clear turn - taking: System initiative should be clear

12) Sense of presence: Users engagement should be as natural as possible

Most of these heuristics state that to increase the usability rating of a virtual reality game or application the movement in the virtual world as well as the character the user takes control of should feel and look as close to real life as possible. Homuncular Flexibility in Virtual Reality\cite{won2015homuncular} explores users controlling bodies different from their own e.g. animals or the human form with extra limbs, these bodies also have completely different controls ''Tracked movements that the user made in the physical world be rendered as different movements of the avatar body.'' They conducted an experiment giving the user a third longer arm controlled by the right arms wrist rotation, the user was then presented with an array of boxes and the aim was to hit the white boxes. The results found users could quickly adapt to use a body that looked and  moved completely different from their own.

\section*{VR usability for educational games}

Maria Virvou and George Katsionis investigated usability issues related with educational virtual reality games\cite{virvouusability}. They conducted their study on 50 children aged 11 - 12 using a VR game in which the player has to answer questions on Geography to progress through the game. After observing the data collected three main usability issues arose; A players understanding of the user interface, the ease of navigation within the virtual world and environmental distraction in game. These three issues were then investigated in how much time was wasted because of these issues, the participants were also split into groups depending on their gaming experience (novice, intermediate and experienced) the study found issues with navigation and user interface got better with gaming experience but in game environmental distraction was not dependant on any level of gaming experience. In conclusion using virtual reality for educational games has no obvious usability problems compared to non - VR educational games.

\section*{Seated vs. Room - Scale VR}

The main disadvantage to both AR and more so VR is that they need to relatively large empty space to use it in, the paper A Comparison of Seated and Room - Scale Virtual Reality in a Serious Game for Epidural Preparation\cite{shewaga2017comparison} discusses the usability of seated VR compared to room - scale that allows the user to walk around. They conducted an experiment of 40 participants either graduate or undergraduate, and used a repeated measures design, meaning each participant tried seated and room - scale with a series of questions after each one. The experiment found that in room - scale VR participants finished tasks quicker and had better precision, also participants responded when asked that they preferred the room - scale more than seated. To relate this to Alistair Sutcliffe and Brian Gaults heuristics tailored for the virtual reality usability, room - scale VR would increase the realism of the scenario by encouraging natural engagement, natural expression of action and sense of presence.

\section*{Conclusion}

The heuristics tailored for virtual reality have been found true for most VR situations to improve usability and therefore enjoyment for the user, but there are acceptations, as the Homuncular Flexibility in Virtual Reality found that the body the user controls does not have to look or move similar to a human.

Sutcliffe and Gaults heuristics are good to take into considerations when designing a AR or VR game but should not hinder the creativeness of the design.

\bibliographystyle{ieeetran}
\bibliography{References}

\end{document}
